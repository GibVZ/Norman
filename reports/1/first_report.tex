\documentclass[a4paper, 12pt]{report}

% Чтобы работала кириллица
\usepackage[T2A]{fontenc}
\usepackage[utf8]{inputenc}

% Делаем человеческие отступы (экономим бумагу)
\usepackage[left=1.3cm,right=1.3cm,top=2cm,bottom=2cm]{geometry}

% Чтобы можно было вставлять кириллицу в формулы
\usepackage{amsmath}

% Вставляем картинки
\usepackage{graphicx}

% Для ссылок во всемирную сеть
\usepackage{nohyperref}  % This makes hyperref commands do nothing without errors
\usepackage{url}  % This makes \url work


% Меняем подпись рисунков и таблиц на русский
\renewcommand{\figurename}{Рис.}
\renewcommand{\tablename}{Табл.}

% Title Page
\title{Отчет о проделанной работе за какую-то там неделю}
\author{Зелинский Виктор}
\date{1 апреля 2024}


\begin{document}
	\maketitle
	\newpage
	Реализован метод Эйлера, взаимодействие между частицами - потенциал Леннарда-Джонса, вместо периодических граничных условий - зеркальное отражение. Визуализация пока все еще в blender.
	
	Для того, чтобы в будущем оценить преимущество метода Верле перед эйлером - посмотрим на границы применимости, проведя моделирования при различных гиперпараметрах.
	
	\begin{figure}[h]
		\includegraphics[width=0.9\linewidth]{big_plot500.png}
		\caption{Графики энергии для различных значений кинетической, потенциальной и суммарной энергией}
		\label{fig:bigplot}
	\end{figure}

	Можно заметить, что для маленьких $\Delta t$  энергия сохраняется с хорошей точностью, вне зависимости от количества частиц(а может, надо было просто диапазон больше взять), но для значения $\Delta t = 5\cdot 10^{-4}$ энергия уже перестает сохранятся.
\end{document}          
